% LaTeX resume using res.cls
\documentclass[margin]{res}
\usepackage{anyfontsize}
\usepackage{t1enc}
\linespread{1.25}
\usepackage{xcolor}
\definecolor{darkblue}{rgb}{0,0,0.5}
\usepackage[colorlinks=true, urlcolor=darkblue]{hyperref}
\usepackage{multicol}
%\hypersetup{colorlinks=true, urlcolor=black}
%\usepackage{helvetica} % uses helvetica postscript font (download helvetica.sty)
%\usepackage{newcent}   % uses new century schoolbook postscript font
\setlength{\textwidth}{5.1in} % set width of text portion
\setlength{\parindent}{0pt}
\newcommand{\forceindent}{\leavevmode{\parindent=1.5em\indent}}

\newcommand{\en}[1]{
\color{[#1]}
}


\begin{document}


\moveleft.5\hoffset\centerline{\Large\bf LINGBO SHEN}
\vspace{-0.2cm}
\moveleft\hoffset\vbox{\hrule width\resumewidth height 1pt}







\begin{resume}
\vspace{-0.9cm}
\section{\textbf{CONTACT}}
\begin{tabular}{p{4.5cm}l}
Koopmans Building   	& Phone (office):	+31 (0)13 466 2043 					\\
Room K642  		&  Phone (mobile) : 	+31 (0)62 747 1549  \\
P.O. box 90153, 5000 LE   & Email: l.shen\_2@tilburguniversity.edu     	\\
Tilburg, the Netherlands  & Website: \href{https://sites.google.com/site/lingboshen1990uvt/}{sites.google.com/site/lingboshen1990uvt/} 

\end{tabular}

\section{\textbf{EDUCATION}}
\textbf{Tilburg University, CentER Graduate School}	\hfill  Tilburg, the Netherlands	\\
	\forceindent Ph.D. in Finance, 2016--2022 (expected) 									\\
	\forceindent Research Master in Finance, 2014--2016 									\\
\textbf{Tilburg University, TiSEM}						 \hfill Tilburg, the Netherlands	\\
	\forceindent M.Sc. in Finance (\textit{cum laude}), 2013--2014										\\
 \textbf{Peking University, Guanghua School of Management}						\hfill  Beijing, China			\\
	\forceindent Bachelor of Economics, majoring in Finance, 2009--2013				\\
	\forceindent Bachelor of Science, dual degree, majoring in statistics, 2010--2013				\\
	\forceindent TiSEM Tilburg University, exchange student, 2011	\hfill  Tilburg, the Netherlands 

\section{\textbf{RESEARCH\ INTERESTS}} Behavioral finance, corporate finance 

\vspace{0.9em}


\section{\textbf{WORKING \ PAPER}} 
\textbf{Homophily in financial market: Evidence from executives and analysts interaction in conference calls} (Job Market Paper)\\  
\textit{Abstract:} I examine whether and how homophily affects executives and analysts interactions in conference calls. Using earnings conference calls of S\&P 1500 firms, I find robust evidence that executive behavior is driven by homophily: they are more likely to select the same ethnic background analysts to ask questions in the conference call. This positive relation is still held by exploiting the exogenous shock to analysts' attention and the event of CEO turnover. Moreover, executives communicate more and speak with a more positive tone to the same ethnic group analysts in the Q\&A session. Such a homophily behavior of information communication has consequences to the analysts forecast quality: their forecast errors become smaller. In summary, my findings suggest that homophily does exist in the financial market and can affect the interactions of different economic agents in the financial market.

\vspace{0.9em}

\textbf{Firms' demands on inventor executives around IPOs} (with Yanying Lyu)\\  
\textit{Abstract:} We examine how going public affects firms' demands on inventor executives who own both  management and innovation experience. Using IPO withdrawn firms as the control group and NASDAQ returns fluctuations as the instrumental variable on IPO competition, we show that firms demand more inventor executives after successful IPOs. We conjecture that firms demand those  inventor executives who own superior abilities to better overcome threats and competitions after going IPOs. Empirical evidence supports this idea. The demand is higher for firms with higher product market competition and innovation competition. In addition, the number of inventor executives is positively related to survival probability after IPOs, stock market performance, operating performance, and innovation performance. Our results provide new evidence on the effect of going public on human capital mobility from firms' demand perspective.

\vspace{0.9em}

\textbf{Team versus individual: Evidence from financial analysts during COVID-19 pandemic} \\  
\textit{Abstract:} Whether the performance of teams is better than that of individuals is still an open question. In this paper I look at one important team in the financial market, the financial analyst team, and take the advantage of the exogenous shock introduced by stay-at-home orders in the United States during the COVID-19 pandemic, to investigate whether analyst teams can issue more timely forecasts compared to individual analysts. The main findings are: (i) after the stay-at-home orders, both team and individual analysts issue fewer timely forecasts; (ii) compared to individual analysts, team analysts can issue more timely forecasts after the stay-at-home orders; and (iii) teams with female leaders and higher diversity perform better. In summary, this paper provides new evidence on the performance difference between teams and individuals especially during tough times.

\section{\textbf{WORK IN \ PROGRESS}} 
\textbf{An empirical test of the wake-up call: Evidence from conference calls
} (with Zilong Niu)\\


\section{\textbf{PRESENT- \ ATIONS}}
\textbf{2021:} Nanjing University$^{\dagger}$; New Zealand Finance Meeting$^{\ast\dagger}$ \\
\textbf{2020:} Southwestern University of Finance and Economics$^{\ast}$  \\
\textbf{2017-2019:} Tilburg University Brown Bag Seminar\\
\textit{$^{\ast}$presented by co-author} \\
\textit{$^{\dagger}$presented virtually}

 

\section{\textbf{HONORS \& AWARDS}}
Tilburg University, CentER scholarship 				\hfill 2014-2016	\\
Peking University, first place of scholarship for first year students\hfill 2009

\section{\textbf{TEACHING \ EXPERIENCE}}
\textbf{Tilburg University}	\\
	\forceindent Supervision B.Sc. Thesis in Finance, 2017    \\
	\textit{Teaching assistant:}\\
	\forceindent Financial History and Intermediation (undergraduate), 2016, 2017, 2018 \\
	\forceindent Finance I for IBA (undergraduate), 2017, 2018, 2019    \\
	\forceindent International Finance (graduate), 2017, 2018, 2019, 2020\\
	\forceindent Academic Competences Finance (pre-master), 2019


\section{\textbf{OTHER \ INFORMATION}}
\textbf{Software skills}: Stata, Python, \LaTeX 	\\
\textbf{Language skills}: English (fluent), Chinese (native), Dutch (A2 level)


\section{\textbf{REFERENCE}}
\vspace{0.1cm} 
\hspace*{-0.3cm}
\begin{tabular}{ll}
	\href{https://www.bwl.uni-mannheim.de/en/spalt/team/prof-dr-oliver-spalt/}{\textbf{Prof. Oliver G. Spalt}} &  {\href{https://www.tilburguniversity.edu/staff/o-wilms}{\textbf{dr. Ole Wilms}}} \\
	Business School & Finance Department \\
	University of Mannheim & Tilburg University \\
	Phone (office): +49 621 181-1518 &  Phone (office): +31 (0)13 466 4821 \\
	Email: spalt@bwl.uni-mannheim.de & Email: o.wilms@tilburguniversity.edu \\
	&  \\
	\href{https://www.wiwi.uni-muenster.de/fcm/en/the-fcm/pf/team/christoph-schneider}{\textbf{Prof. Christoph Schneider}} &  \\
	School of Business and Economics &  \\
	University of Muenster  &  \\
	Phone (office): +49 251 83-22088  &  \\
	Email: christoph.schneider@wiwi.uni-muenster.de &  \\
\end{tabular}%

 

 
\begin{flushright}
	Updated: 2022 January
\end{flushright}
 
\end{resume}


\end{document}




\textbf{The effect of diversity on firm actions in tough times} (with Alberto Manconi, A. Emanuele Rizzo, Christoph Schneider, and Oliver Spalt)\\
\\
\textit{Abstract:} We investigate the effect of top management team diversity on firm actions and performance. Using tariff reductions as an exogenous shock to competitive pressure for U.S. manufacturing industry companies and building a novel text-based top management team diversity measure, we show that team diversity is related to firm policies and performance after the tariff reduction shocks. Specifically, firms with more diversified top
management teams have higher ROA and Market-to-Book, cut more investments in the affected segments, and reallocate more investment to unaffected segments within the firms. Our results support the argument that diversity of top managers is benefit to firms during tough times.

\textit{Abstract:} By exploiting the pass of Germany Federal Act on Gender Equality in 2015, we investigate its effect on gender pay gap. The law requires equality between women and men in the federal administration and in federal enterprises and courts. Using detailed German employee-employer administration data a difference-in-differences approach, we find that the law reduces the gender pay gap, primarily by slowing the wage growth for male employees. In addition, the effect is more pronounced on high wage proportion employees and employees with high level working skills. 